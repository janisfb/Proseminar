\chapter{Einleitung}

\section{Motivation}
Mit der vierten industriellen Revolution erhöht sich auch die Anzahl der Arbeitsplätze, an denen Mensch und Maschine ohne physikalische Barrieren zusammen arbeiten. Robotersysteme wie der \textit{LBR iiwa}\footnote{\url{https://www.kuka.com/de-de/future-production/mensch-roboter-kollaboration}~(besucht am 14.12.2019)} von der Firma Kuka sind ausgelegt für diese Art von Zusammenarbeit. Die Aufgabenbereiche sind oft komplex und erfordern Geschwindigkeit und Präzision. Eine klassische Trennung zwischen automatisierten und manuellen Arbeitsplätzen ist deswegen oft nicht mehr möglich.~\cite{ObererTreitz.2019} Höchste Priorität bei dieser Kollaboration zwischen Mensch und Roboter hat die Sicherheit des Menschen. Zur Vermeidung von Unfällen benötigt der Bediener zu jedem Zeitpunkt volle Kontrolle. Die Entwicklung einer direkten natürlichen Kommunikationsmöglichkeit mit den Maschinen könnte dazu beitragen. So wäre es möglich, einen laufenden Prozess im Notfall auch ohne den umständlichen Weg über die Bedieneinheit zu stoppen. Nützlich wäre es, wenn der Roboter menschliche Gesten erkennen würde. Erste vielversprechende Forschungen im diesem Bereich eröffnen neue Möglichkeiten, bei denen Roboter auf statische Gesten reagieren können~\cite{flexibleSystem}.

Industrieroboter werden schon seit Jahrzehnten weltweit in der Massenproduktion eingesetzt~\cite{ObererTreitz.2019}. Die Roboter führen Aufgaben automatisiert durch und erhöhen die Effizienz von Konstruktionsstätten erheblich. Gerade bei der klassischen Serienproduktion, wie zum Beispiel in der Automobilindustrie, ist eine konkurrenzfähige Fabrik nicht mehr ohne Industrieroboter möglich. Diese Systeme stoßen an ihre Grenzen, wenn eine Veränderung im Produktionsprozess auftritt. Für Kleinserien ist eine vollautomatisierte Produktion oft nicht wirtschaftlich, da die Anschaffung von Sensorsystemen oder die häufige Neukalibrierung zu kostenintensiv sind. Robotersysteme, die anpassungsfähig sind und bei minimalen Veränderungen im Ablauf keine Neuentwicklung benötigen, sind also durch herkömmliche Systeme nicht möglich. Mit der Einführung der Mensch-Roboter-Kollaboration konnte genau auf diese Probleme eingegangen werden. Die Grundidee einer solchen Zusammenarbeit zwischen Mensch und Maschine ist, dass beide Parteien ihre jeweiligen Stärken nutzen und so das Einsatzspektrum von Robotern deutlich erhöhen können.~\cite{ObererTreitz.2019} Beim Menschen sind beispielsweise die Auge-Hand-Koordination oder die Fähigkeit zur selbstständigen Problemlösung besonders ausgeprägt. Ein Roboter beherrscht Präzision und Ermüdungsfreiheit deutlich besser.~\cite{fraunhoferMRK}

\section{Gründe für den Einsatz von Gestensteuerung}
Gerade weil Mensch-Roboter-Kollaboration einen gemeinsamen Arbeitsraum voraussetzen -- also ein Einsatz von Schutzeinrichtungen wie Zäunen nicht möglich ist -- müssen, besonders in der industriellen Produktion, Systeme zum Einsatz kommen, die eine sichere Bedienung ermöglichen. Aber auch beim Einsatz von kraft- und leistungsreduzierten Robotern, bei denen auch im Falle eines Zusammenstoßes keine Verletzungen entstehen, müssen kollaborative Roboter über Sensorik verfügen, die eine direkte Bedienung ermöglicht. Schon jetzt werden Roboter in Anwendungsgebieten wie der Altenpflege genutzt und treffen dabei auch auf ungeschulte Personen.~\cite{fraunhoferMRK} Eine komplizierte Bedienung über ein klassisches Eingabesystem ist also ausgeschlossen. Vielmehr müssen Systeme eingesetzt werden, die auch natürliche Kommunikationsversuche wie Sprache und Gestik erkennen und entsprechend reagieren. In dieser Arbeit sollen die Gestik und die damit verbundene Gestenerkennung genauer betrachtet werden.

Die Gestenerkennung ist eine Fachrichtung, die neben der Forschung auch in der Industrie immer größere Aufmerksamkeit genießt. Die Anwendungsgebiete erstrecken sich über ein breites Spektrum und umfassen zum Beispiel die Entwicklung von Hilfssystemen für Sprachgeschädigte, die Manipulation von virtuellen Umgebungen oder auch die Erkennung von Müdigkeit bei Autofahrern~\cite{recognitionSurvey}. Die Gestik gehört zu den natürlichen Kommunikationsmitteln des Menschen und ist damit als Bedienungsschnittstelle für Roboter besonders gut geeignet. Ebenfalls wird die Gestenerkennung nicht wie die Spracherkennung von Maschinengeräuschen beeinflusst, die in Fabriken oft nicht verringert werden können.
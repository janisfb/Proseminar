\chapter{Fazit und Ausblick}
Damit Unfälle mit kollaborativen Robotern irgendwann ausgeschlossen werden können, muss auch in Zukunft der Fokus von Forschung und Industrie auf der Entwicklung von Systemen zum Schutz des Menschen liegen. Mittels Sensorik müssen dabei die Steuerung und selbstständige Erkennung von Gefahren verbessert werden. Bekanntere Systeme, wie die Spracherkennung, sind in lauter Arbeitsumgebung unbrauchbar. Deshalb wurde in dieser Arbeit eine Möglichkeit zur Interaktion mit kollaborativen Robotern mittels Gestik vorgestellt. Die einfache und schnelle Form der Bedienung bringt deutliche Vorteile für die Mensch-Roboter-Kollaboration. Gesten sind natürliche Mittel der Kommunikation und gehören damit zu den besonders gut geeigneten Steuerungstechniken. In den letzten Jahren werden Roboter vermehrt auch in Umgebungen eingesetzt, in denen es kein geschultes Personal gibt. Die Bedienung auf Basis von natürlichen Kommunikationsmitteln wie Sprache und Gestik wird also in Zukunft immer wichtiger werden. 

Die in dieser Arbeit vorgestellte Methode ist eine kamerabasierte Gestenerkennung. Aus den Daten einer RGB-D Kamera kann eine vorher definierte Geste erkannt werden. Die Klassifizierung und der Trainingsprozess für das System basieren auf dem von Nicholas Gillian entwickelten ANBC-Algorithmus~\cite{gillianANBC}. Bei der Auswertung konnte festgestellt werden, dass es sich bei dieser Zusammenstellung um ein robustes System handelt. In Einsatzszenarien mit geringer Komplexität kann das System für einfache Befehle schon verwendet werden.

Allerdings werden Roboter in vielen unterschiedlichen Anwendungsgebieten eingesetzt. Ein flexibles System der Gestenerkennung sollte in jedem dieser Einsatzszenarien mit möglichst geringen Anpassungen funktionieren. Ein Problem des in dieser Arbeit vorgestellten Systems ist, dass die Erkennung nur mit relativ geringer Entfernung zuverlässige Ergebnisse liefert. Weiterhin konnte festgestellt werden, dass der Einsatz bei mobilen Robotern zu Schwierigkeiten führen könnte.

Das vorgestellte System bietet also eine gute Basis, bei der alle Grundfunktionen schon einsetzbar sind. Aufgrund der immer komplexer werdenden Anforderungen in der Mensch-Maschine-Interaktion müssen die Systeme weiter angepasst und perfektioniert werden. In Zukunft sollte auf diesen Ergebnissen aufgebaut werden und die Flexibilität und damit auch die Einsetzbarkeit erhöht werden. Möglichkeiten dazu sind zum Beispiel die Datengewinnung mit Hilfe von mehreren Kameras~\cite{multipleDepthCameras} oder die Erkennung von dynamischen Gesten~\cite{hiddenMarkov}. Generell wird die stetige technische Innovation dazu führen, dass sich das Anwendungsspektrum von Robotern deutlich erweitert. Klassische Eingabesysteme wie die Tastatur stoßen dabei schnell an ihre Grenzen und müssen durch neue Interaktionsformen ersetzt werden. Die Interaktion per Geste ist eine zukunftsfähige Technik, die eine immer größer werdende Rolle spielen wird.  

 
\chapter{Evaluation des vorgestellten Systems}

\section{Mögliche Implementierung und Test der Erkennungsrate}
In~\cite{flexibleSystem} wurde ein System entwickelt, dass intern genau die Komponenten und Methoden nutzt, die in Kapitel~\ref{Gestenerkennungskapitel} erarbeitet wurden. Es wird dargestellt, wie es möglich ist, eine Mensch-Roboter-Kollaboration mit Erkennung von statischen Gesten umzusetzen. Zur Veranschaulichung der Entwicklungen wurden Gesten für das Stoppen und Fortfahren eines Roboters definiert. Dabei gibt es eine Stop- und Fortfahren-Geste pro Körperhälfte. Implementiert wurde das System unter Benutzung des speziell für die Entwicklung von Software für Roboter bereitgestellten Robot Operating System (ROS)\footnote{\url{https://www.ros.org/}}. Das ROS-Framework beinhaltet eine Vielzahl an Gerätetreibern und Softwarebibliotheken und kann deswegen in den unterschiedlichsten Robotersystemen mit relativ geringem Aufwand eingesetzt werden. 

Der Trainingsdatensatz wurde zusammengestellt, in dem von vier Benutzern (2 Frauen, 2 Männer) jeweils 200 Gestenvektoren pro zu erkennender Gesten aufgenommen und entsprechend gekennzeichnet wurden. Damit festgestellt werden kann, wie der Trainingsdatensatz für ein erfolgreiches Lernen aufgebaut sein muss, wurde das Training mit unterschiedlichen Teilmengen durchgeführt. Dabei wurde die k-fache Kreuzvalidierung verwendet, um zu evaluieren, ob das Modell die Muster in den Daten korrekt aufgenommen hat. Der jeweils genutzte Datensatz wurde zu 85 Prozent für das Training und zu 15 Prozent für die Validierung aufgeteilt. Das Ergebnis von diesem Test war, dass es nötig ist das System mit Menschen von unterschiedlicher Größe, Körperform und Geschlecht zu trainieren. Beispielsweise konnte eine Überanpassung festgestellt werden, wenn das System nur mit Trainingsdaten von Männern trainiert wurde und eine Geste von einer Frau erkannt werden sollte. Die Nutzung des gesamten 3200 Gestenvektoren umfassenden Trainingssatzes erwies sich als geeignet, um eine robuste Erkennung für viele unterschiedliche Nutzer und Variabilitäten in der Gestenausführung zu erreichen.~\cite{flexibleSystem}

Um die Performance unter realitätsnahen Umständen zu testen, wurde ein Test mit unterschiedliche Nutzern ohne Vorerfahrungen mit dem System durchgeführt. Dafür wurde der Kinect auf Hüfthöhe angebracht und die Tester gebeten eine vordefinierte Reihenfolge an Gesten mit unterschiedlichen Positionen und Entfernungen auszuführen. Die Erkennungsrate lag bei 96,7\% für alle durchgeführten Tests. Die Genauigkeit der Stop-Geste war 100\%. Das System bei den weiter entfernten Positionen (2 und 5 Meter) eine bessere Trefferrate als bei der geringeren Entfernung (0,5 Meter). Die Performance war schlechter für Positionen, bei denen die Tester nicht frontal zum Sensor standen. Außerdem traten keine Fälle auf, in denen eine Stop-Geste erkannt wurde, obwohl eine Fortfahren-Gesten durchgeführt wurde oder eine Fortfahren-Geste erkannt wurde, obwohl eine Stop-Geste durchgeführt wurde.~\cite{flexibleSystem}

\section{Bewertung des Systems und der Einsetzbarkeit}
Insgesamt können mit einem System, dass den Kinect-Sensor für die Datensammlung und den ANBC-Algorithmus für die Erkennung der Gesten nutzt sehr gute Ergebnisse erzielt werden. Die Erkennung ist robust und funktioniert auch bei großer Variabilität in der Ausführung ohne große Probleme. Allerdings wurde das System noch nicht ausreichend in der industriellen Anwendung getestet. Der Kinect-Sensor ist mit seiner geringen Reichweite für viele Robotersysteme nur bedingt geeignet. Ein Roboter, dem ein mobiles Gefährt zur Verfügung steht würde zum Beispiel immer wieder Kalibrierungsproblemen im Kinect-Sensor hervorrufen, da dieser bei Bewegung die Distanzen nicht mehr korrekt berechnen kann. Außerdem ist es zu erwarten, dass ein System, das nur einen Sensor für die Datensammlung nutzt verdeckt werden könnte, was eine Gestenerkennung unmöglich machen würde. Ein System mit mehreren Tiefenkameras wurde zum Beispiel in~\cite{multipleDepthCameras} vorgestellt und hat den Vorteil, dass der Nutzer auch \SI{360}{\degree} Bewegungen ausführen kann ohne dabei eine Verdeckung durch den Torso befürchten zu müssen.

Für Roboteranlangen gilt generell ein besonders hohes Sicherheitsniveau, das natürlich auch für die Gestenerkennung gelten muss. Die Anlagen müssen dabei so ausgelegt sein, dass eine möglichst hohe Produktivität der Roboter erreicht wird, bei der der Bedienerschutz trotzdem noch vollständig gegeben ist. Die Systemleistung wird dabei besonders gut ausgenutzt, wenn der Roboter unter hoher Geschwindigkeit und Stärke arbeitet. Latenzzeiten in der Gestenerkennung müssen also sehr gering sein, damit ein Einsatz in der Industrie unbedingt möglich ist. Außerdem ist davon auszugehen, dass die Erkennungsrate für eine Genehmigung nach DIN-Normen (z.B. DIN EN ISO 12100:2011-03 2011), die Voraussetzungen für den Betrieb einer Anlage sind, so perfekt sein muss, dass der Nutzer zu jedem Zeitpunkt und in jeder Situation Kommandos ausführen kann.~\cite{ObererTreitz.2019}

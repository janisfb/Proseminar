\chapter{Einleitung}

\section{Motivation}
Mit der vierten industriellen Revolution erhöht sich auch die Anzahl der Arbeitsplätze, an denen Mensch und Maschine ohne physikalische Barrieren zusammen arbeiten. Dabei sind die Aufgabenbereiche oft komplex und erfordern Geschwindigkeit und Präzision. Bei der Umsetzung von Produktionsstätten, die auf einer Kollaboration zwischen Mensch und Roboter basieren, hat die Sicherheit des Menschens immer höchste Priorität. Zur Vermeidung von Unfällen, ist es wichtig, dass der Bediener zu jedem Zeitpunkt volle Kontrolle über den Roboter hat. Eine Möglichkeit dazu ist, eine direkte natürliche Kommunikationsmöglichkeit mit den Maschinen zu entwickeln. Dabei könnten wie in~\cite{flexibleSystem} zum Beispiel Gesten erkannt werden. So wäre es möglich, einen laufenden Prozess per Stop-Geste zu unterbrechen, ohne vorher den unter Umständen weiten Weg zur Bedieneinheit machen zu müssen.   

\section{Einführung in die Mensch-Roboter-Kollaboration}
Industrieroboter werden schon seit Jahrzehnten weltweit in der Massenproduktion eingesetzt. Die Roboter führen Aufgaben automatisiert durch und erhöhen die Effizienz von Konstruktionsstätten erheblich. Gerade bei der klassischen Serienproduktion, wie zum Beispiel in der Automobilindustrie, ist eine konkurrenzfähige Fabrik nicht mehr ohne Industrieroboter möglich. Allerdings kommen diese Systeme an ihre Grenzen, wenn es eine Veränderung im Produktionsprozess gibt. Für Kleinserien ist eine vollautomatisierte Produktion oft nicht wirtschaftlich, da die Anschaffung von Sensorsystemen oder die häufige Neukalibrierung zu kostenintensiv sind. Robotersysteme, die anpassungsfähig sind und bei minimalen Veränderungen im Ablauf keine Neuentwicklung benötigen, sind also durch herkömmliche Systeme nicht möglich. Mit der Einführung der Mensch-Roboter-Kollaboration konnte genau auf diese Probleme eingegangen werden. Die Grundidee einer solchen Zusammenarbeit zwischen Mensch und Maschine ist, dass beide Parteien ihre jeweiligen Stärken nutzen und so das Einsatzspektrum von Robotern deutlich erhöhen können.~\cite{ObererTreitz.2019} Beim Menschen sind beispielsweise die Auge-Hand-Koordination oder die Fähigkeit zur selbstständigen Problemlösung besonders ausgeprägt. Ein Roboter beherrscht Präzision und Ermüdungsfreiheit deutlich besser.~\cite{fraunhoferMRK}

\section{Gründe für den Einsatz von Gestensteuerung}
Gerade weil Mensch-Roboter-Kollaboration einen gemeinsamen Arbeitsraum voraussetzen -- also ein Einsatz von Schutzeinrichtungen wie Zäunen nicht möglich ist -- müssen, besonders in der industriellen Produktion, Systeme zum Einsatz kommen, die eine sichere Bedienung ermöglichen. Aber auch beim Einsatz von kraft- und leistungsreduzierten Robotern, bei denen auch im Falle eines Zusammenstoßes keine Verletzungen entstehen, müssen kollaborative Roboter über Sensorik verfügen, die eine direkte Bedienung ermöglicht. Schon jetzt werden Roboter in Anwendungsgebieten wie der Altenpflege genutzt und treffen dabei auch auf ungeschulte Personen.~\cite{fraunhoferMRK} Eine komplizierte Bedienung über ein klassisches Eingabesystem ist also ausgeschlossen. Vielmehr müssen Systeme eingesetzt werden, die auch natürliche Kommunikationsversuche wie Sprache und Gestik erkennen und entsprechend reagieren. In dieser Arbeit sollen die Gestik und die damit verbundene Gestenerkennung genauer betrachtet werden.

Die Gestenerkennung ist eine Fachrichtung, die neben der Forschung auch in der Industrie immer größere Aufmerksamkeit genießt. Die Anwendungsgebiete erstrecken sich über ein breites Spektrum und umfassen zum Beispiel die Entwicklung von Hilfssystemen für Sprachgeschädigte, die Manipulation von virtuellen Umgebungen oder auch die Erkennung von Müdigkeit bei Autofahrern. Die Gestik gehört zu den natürlichen Kommunikationsmitteln des Menschen und ist damit als Bedienungsschnittstelle für Roboter besonders gut geeignet. Ebenfalls wird die Gestenerkennung nicht wie die Spracherkennung von Maschinengeräuschen, die in Fabriken oft nicht verringert werden können, beeinflusst.